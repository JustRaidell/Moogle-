\documentclass[12pt]{article}
\begin{document}

\textbf {Moogle!} es un programa que hace b\'usquedas en tiempo real sobre una base de datos de documentos.
En principio necesita una base de datos de archivos en formato .txt ubicados en la carpeta Content.
Cuenta adem\'as del archivo principal Moogle.cs con 5 archivos de extensi\'on .cs necesarios (bueno 4 de ellos necesarios para funcionar)
que son:
\begin{itemize}
    \item AuxItems, una clase que guarda el texto y t\'itulo de un documento.
    \item Methods, una clase que tiene todos los metodos que son invocados en el programa.
    \item SearchItem, una clase que guarda el t\'itulo, snippet y score de cada documento al hacer la query
    \item SearchResult, que como su nombre indica devuelve el resultado de la b\'usqueda.
    \item Matrix, una clase que ayuda en el trabajo con operaciones con matrices, pero que no fue necesario usar en el funcionamiento del proyecto.
\end{itemize}

Lo primero que hace el proyecto es hacer un arreglo de AuxItems donde en cada posici\'on hay un documento de la
carpeta Content.
Luego duplica ese arreglo pero con los documentos normalizados, esto es, los mismos textos de los documentos,
pero sin signos de puntuaci\'on, may\'usculas ni saltos de l\'inea.
Ahora creo un diccionario y voy recorriendo los textos de mis documentos. Cada vez que encuentro una palabra
"nueva" para mi diccionario la agrego con un n\'umero natural que significa que n\'umero de palabra es. Paralelo a
esto voy creando una matriz de (palabras ; documentos). Al terminar esta parte del c\'odigo me queda un diccionario
en el que a todas las palabras presentes se les asigna un n\'umero, y una matriz de tama\~no [cant de palabras ; cant de
documentos]

\begin {center}
\begin{tabular}{c|cccc}
     & 1\tiny er \small documento & 2\tiny do \small documento & ... & n-\'esimo documento \\
     \hline
     1\tiny ra \small palabra &  &  &  &  \\
     2\tiny da \small palabra &  &  &  &  \\
     3\tiny ra \small palabra &  &  &  &  \\
     ... &  &  &  &  \\
     m-\'esima palabra &  &  &  &  \\
\end{tabular}
\end {center}

Aqu\'i empieza la parte algebraica, pues por cada documento voy calculando la importancia que tiene cada una de
las palabras que lo forman, a trav\'es de un m\'etodo llamado $TF*IDF$. Este m\'etodo toma la cantidad de veces 
que se repite una palabra en un documento (Term Frequency o \emph{TF}) y lo multiplica por la "rareza" de la palabra 
(Inverse Document Frequency o \emph{IDF}) para as\'i obtener la relevancia de esa palabra para cada documento. En este 
punto ya tengo el IDF de cada palabra de mi universo ya calculada (ya que este no var\'ia) y solo llamo al m\'etodo privado 
TF que calcula el TF de cada una de las palabras de un documento. Al multiplicarlas por su respectivo IDF
me queda la importancia de esas palabras para ese documento. Luego le asigno ese valor a su posici\'on correspondiente 
en la matriz. Al acabar me queda la matriz con el $TF*IDF$ de cada palabra de mi universo de palabras en cada 
documento (el equivalente a la importancia de cada palabra para cada documento).

En este punto del programa es que valoro la query. Lo primero que hago es normalizarla. Luego le calculo el
$TF*IDF$ para ver qu\'e tan relevante es para mi universo. En caso de que no tenga palabras en com\'un con mis
documentos su relevancia va a ser obviamente 0, y a partir de ah\'i se incrementar\'a en dependencia.
Una vez que tengo el $TF*IDF$ de la query puedo aplicar una f\'ormula llamada "Similitud Coseno" que me dice el
\'angulo de desviaci\'on de un vector respecto a otro. Si ese coseno da 0 entonces ambos vectores apuntan al mismo
lugar.

Aplicando esto al proyecto. Si trato los documentos como un conjunto de vectores y la query como otro, la similitud
de coseno me dir\'a qu\'e tan parecidos son. Dicho y hecho: creo un m\'etodo que calcula la similitud coseno entre dos
vectores y lo llamo tantas veces como documentos tenga. Me queda un array de valores, los scores: a mayor score,
m\'as parecido es ese documento a la query.

A continuaci\'on saco los Snippet de cada documento. Esto es un fragmento de documento que coincida
parcialmente con la b\'usqueda. Para hacer esto simplamente separo los textos de los documentos por oraciones y
voy guardando en un Array la que mayor cantidad de coincidencias tenga con la query por cada documento.


Para terminar ordeno con un Array de Tuplas los documentos por valor de score y luego creo un array de SearchItem del
mismo tama\~no que la cantidad de documentos con score distinto de 0. En cada posici\'on guardo el t\'itulo, Snippet
sin normalizar y score del documento en orden de importancia descendente. Como resultado todos los documentos
mostrados tienen aunque sea un m\'inimo de coincidencia con la b\'usqueda.

Al final SearchResult con mi array de SearchItem como argumento imprime en pantalla el T\'itulo y el fragmento de
texto de los documentos ordenados de m\'as a menos por valor de score.

\end{document}
