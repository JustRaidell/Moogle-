\documentclass{slides} %What if?
\usepackage{graphicx}


\begin{document}

\title {\textbf {Moogle!}}
\author {Raidel M. Cabellud Lizaso}
\date {23 de julio, 2023}
\maketitle


\begin {center}
\scalebox{.60}{
\begin{tabular}{c|cccc}
     & 1\tiny er \small documento & 2\tiny do \small documento & ... & n-\'esimo documento \\
     \hline
     1\tiny ra \small palabra &  &  &  &  \\
     2\tiny da \small palabra &  &  &  &  \\
     3\tiny ra \small palabra &  &  &  &  \\
     ... &  &  &  &  \\
     m-\'esima palabra &  &  &  &  \\
\end{tabular}
}

\vspace{1cm}

Tabla que ilustra la matriz de $TF*IDF$

\end {center}

\newpage

\begin{flushleft}
     $$TF = \texttt{cantidad de veces que se repite la palabra}$$  
     
     $$IDF = \log{\frac{\small\texttt{cantidad total de documentos}}{\small\texttt{cantidad de documentos que contienen la palabra}}}$$
\end{flushleft}


\newpage

%formula de la similitud de cosenos

La similitud de cosenos responde a la siguiente f\'ormula

\begin{equation}
     \small
     Cs = \frac{a_1b_1+a_2b_2+a_3b_3+...+a_nb_n}{\sqrt{a_1^2+a_2^2+a_3^2+...+a_n^2}*\sqrt{b_1^2+b_2^2+b_3^2+...+b_n^2}}     
\end{equation}

donde $a_1, a_2, a_3,...,a_n$ son las componentes del vector a y an\'alogamente con las componentes del vector b



\end{document}